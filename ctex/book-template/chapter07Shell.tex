\chapter{Bash脚本}
\label{sec:shellScript}
\index{bash}

无聊地重复一件事确实惹人厌!当然了,数次重复一些命令是不是也是很不爽的
一件事呢?如果是请跟随我往下看。当我们熟悉了操作系统后,以期望有更好的
提高,那就是写脚本了。

\section{语法介绍}

\subsection{变量定义}

\subsection{特殊变量}

在脚本里有以下几个特殊的变量,如下

\begin{table}[h]
  \centering
  \begin{tabular}{c|l}
    \hline
    \$0   & 代表该脚本的名字 \\
    \hline
    \$n   & 代表该程序的第n个参数值,n=1..9 \\
    \hline
    \$*   & 代表该脚本的所有参数 \\
    \hline
    \$\#  & 代表该脚本的参数个数 \\
    \hline
    \$\$  & 代表该脚本的PID \\
    \hline
    \$!   & 代表上一个指令的PID \\
    \hline
    \$?   & 代表上一个指令的返回值 \\
    \hline
  \end{tabular}
\end{table}

\subsection{变量赋值和替换}

\subsection{本地变量与全局变量}

\subsection{引用变量}

\subsection{数组}

Bash只支持一维数组,但参数个数没有限制。

数组赋值:

\begin{verbatim}
1. array=(var1 var2 var3 ... varN)
2. array=([0]=var1 [1]=var2 [2]=var3 ... [n]=varN)
3. array[0]=var1
   array[1]=var2
   array[2]=var3
   ...
   array[N]=varN
\end{verbatim}

\begin{verbatim}
  declare -

  MYARRAY=([0]=tom [1]=laven [2]=liu [5]=jim)
  
  数组的元素的长度:
  ${#MYARRAY} 指的是数组MYARRAY的第0个元素的长度,与
  ${#MYARRAY[0]} 等价
  ${#MYARRAY[n]} 是第n+1个元素,

  数组的元素个数:
  ${#MYARRAY[*]}
  ${#MYARRAY[@]}

  一个例子,随机生成10个整型元素的的数组,并找出该数组中的最大值。
  #!/bin/bash
  #
  for i in {0..9}
  do
  ARRAY[$i]=$RANDOM
  echo -n "${ARRAY[$i]} "
  sleep 1
  done

  echo

  declare -i MAX=${ARRAY[0]}
  INDEX=$[${#ARRAY[*]-1}]
  for i in `seq 1 $INDEX`
  do
      if [ $MAX -lt ${ARRAY[$i]} ] ; then
          MAX=${ARRAY[$i]}
      fi
  done

  echo ${MAX}
\end{verbatim}

\subsection{特殊字符}

\section{基本流程}

Bash与其他编程语言一样,也有自己的程序处理逻辑。接下来的这个章节主要介
绍Bash的脚本几种基本流程。

\subsection{if结构}

先看一下bash的man page是如何定义if结构的,

\begin{verbatim}
if list; then list; [ elif list; then list; ] ... [ else list; ] fi
\end{verbatim}

当\emph{if list}执行成功并且返回状态是0时,相应的\emph{then list}就会被
执行;否则,\emph{elif list}执行,且返回状态为0时,相应的\emph{then
  list}就会被执行;之后命令执行结束。如果前面的\emph{if list}及
\emph{elif list}都不能成功执行,那么将执行最后一个\emph{else list}语句。
返回状态就是上一条命令执行成功与否,执行成功就返回0,不成功返回非0。不
成功的原因有很多,成功返回就一个。

一个示例:

\begin{lstlisting}
  #!/bin/bash
  #: Title		: xxxx
  #: Date		: 2012-10-12
  #: Author		: "Laven Liu" ldczz2008@sina.com
  #: Version		: 1.0
  #: Description	: get the maximum number
  #: Options		: None

  NUM=1

  if [ $NUM -gt 0 ]; then
  echo "NUM is great than 0"
  else
  echo "NUM is less than 0"
  fi
\end{lstlisting}

\subsection{for结构}

\begin{lstlisting}
#!/bin/sh

# numberlines - A simple alternative to cat -n, etc.

for filename
do
    linecount="1"
    while read line
    do
        echo "${linecount}: $line"
        linecount="$(($linecount + 1))"
    done < $filename
done
exit 0
\end{lstlisting}

\subsection{while结构}

\begin{lstlisting}
  #!/bin/bash
  
  # Written By: LavenLiu
  # Date: 2014-02-13
  # Mail:  air.man.six@gmail.com
  
  # server_list array saved all the servers' ip addresses
  server_list=(90 110 225 231 233 235 249 251 252)
  
  # how_many is the quantity of the servers
  how_many=$(( ${#server_list[@]} - 1 ))
  
  myloop=0
  
  while [ ${myloop} -lt ${how_many} ] ;
  do
  ping -c1 192.168.18.${server_list[${myloop}]} &> /dev/null
  if [ "$?" -eq 0 ] ; then
  echo "Server 192.168.18.${server_list[${myloop}]} is alive"
  else
  echo "Server 18.${server_list[${myloop}]} is down! Contact the adminstrator..."
  fi
  let "myloop = ${myloop} + 1"
  done
\end{lstlisting}

\section{操作字符串}

\section{函数}

\section{信号捕捉}

\begin{verbatim}
trap可以捕捉信号,根据捕捉到的信号,执行响应的操作。
语法:
trap 'action' SIGNAL
\end{verbatim}

\section{开机脚本启动顺序}

\section{一个实例}

\begin{lstlisting}
  #!/bin/bash
  # Written By: LavenLiu
  # Date: 2013-01-08
  def_colors () {
    # Attributes
    normal='\033[0m';bold='\033[1m';dim='\033[2m';under='\033[4m'
    italic='033[3m'; notalic='\033[23m'; blink='\033[5m';
    reverse='\033[7m'; conceal='\033[8m'; nobold='\033[22m';
    nounder='\033[24m'; noblink='\033[25m'
    
    # Foreground
    black='\033[30m'; red='\033[31m'; green='\033[32m'; yellow='\033[33m'
    blue='\033[34m'; magenta='\033[35m'; cyan='\033[36m'; white='\033[37m'

    # Background
    bblack='\033[40m'; bred='\033[41m'
    bgreen='\033[42m'; byellow='\033[43m'
    bblue='\033[44m'; bmagenta='\033[45m'
    bcyan='\033[46m'; bwhite='\033[47m'
  }

  def_colors
  clear
  hostname=`cat /proc/sys/kernel/hostname`
  echo
  echo -e "System Report for $white$hostname$normal on `date`"
  echo
  processor=`grep 'model name' /proc/cpuinfo | cut -d: -f2 | cut -c2-`
  nisdomain=`cat /proc/sys/kernel/domainname`
  cache=`grep 'cache size' /proc/cpuinfo | awk '{print $4,$5}'`
  bogomips=`grep 'bogomips' /proc/cpuinfo | awk '{print $3}'`
  vendor=`grep 'vendor_id' /proc/cpuinfo`
  echo -e "Hostname: $white$hostname$normal NIS Domain: $white$nisdomain$normal"
  if [ "`echo $vedner | grep -i intel`" ]
  then
     cpu_color=$blue
  elif [ "`echo $vender | grep -i amd`" ]
  then
     cpu_color=$green
  fi
  
  echo -e "Processor: $cpu_color$processor$normal"
  echo -e "Running at $white$bogomips$normal bogomips with \
  $white$cache$normal cache"
  echo
  osrelease=`cat /proc/sys/kernel/osrelease`
  rev=`cat /proc/sys/kernel/version | awk '{print $1}'`
  da_date=`cat /proc/sys/kernel/version | cut -d\  -f2-`
  upsec=`awk '{print $1}' /proc/uptime`
  uptime=`echo "scale=2;$upsec/86400" | bc`
  echo -e "OS Type: $white$ostype$normal Kernel: \
  $white$osrelease$normal"
  echo -e "Kernel Compile $white$rev$normal on \
  $white$da_date$normal"
  echo -e "Uptime: $magenta$uptime$normal days"
  set `grep MemTotal /proc/meminfo`
  tot_mem=$2; tot_mem_unit=$3
  set 'grep MemFree /proc/meminfo'
  free_mem=$2; fre_mem_unit=$3
  perc_mem_used=$((100-(100*free_mem/tot_mem)))
  set `grep SwapTotal /proc/meminfo`
  tot_swap=$2; tot_swap_unit=$3
  perc_swap_used=$((100-(100*free_swap/tot_swap)))
  if [ $perc_mem_used -lt 80 ]
  then
     mem_color=$green
  elif [ $perc_mem_used -ge 80 -a $perc_mem_used -lt 90 ]
  then
     mem_color=$yellow
  else
     mem_color=$red
  fi
  if [ $perc_swap_used -lt 80 ]
  then
     swap_color=$green
  elif [ $perc_swap_used -ge 80 -a $perc_swap_used -lt 90 ]
  then
     swap_color=$yellow
  else
     swap_color=$red
  fi
  
  echo -e "Memory: $white$tot_mem$normal $tot_mem_unit Free: $white$free_mem$normal \
  $fre_mem_unit $Used: $mem_color$perc_mem_used$normal"
  echo -e "Swap: $white$tot_swap$normal $tot_swap_unit Free: $white$free_swap$normal \
  $fre_swap_unit $Used: $swap_color$perc_swap_used$normal"
  echo
  set `cat /proc/loadavg`
  one_min=$1
  five_min=$2
  fifteen_min=$3
  echo -n "Load Avderage:"
  for ave in $one_min $five_min $fifteen_min
  do
     int_ave=`echo $ave | cut -d. -f1`
     if [ $int_ave -lt 1 ] ; then
        echo -en "$green$ave$normal "
     elif [ $int_ave -ge 1 -a $int_ave -lt 5 ] ; then
        echo -en "$yellow$ave$normal "
     else
        echo -en "$red$ave$normal"
     fi
  done
  echo
  running=0; sleeping=0; stopped=0; zombie=0
  for pid in /proc/[1-9]*
  do
     procs=$((procs+1))
     stat=`awk '{print $3}' $pid/stat`
     case $stat in
     R) running=$((running+1));;
     S) sleeping=$((sleeping+1));;
     T) stopped=$((stopped+1));;
     Z) zombie=$((zombie+1));;
     esac
  done
  echo -n "Process Count: "
  echo -e "$white$process$normal total $white$running$normal running \
  $white$sleep$normal sleeping $white$stopped$normal stopped \
  $white$zombie$normal zombie"
  echo
\end{lstlisting}

下面是该脚本的输出结果:

\small{
\begin{verbatim}
System Report for richard on Tue Sep 23 21:34:22 CST 2014

Hostname: richard NIS Domain: (none)
Processor: 
Running at 4585.16
4585.16
4585.16
4585.16 bogomips with 3072 KB
3072 KB
3072 KB
3072 KB cache

OS Type: Linux Kernel:3.13.0-35-generic
Kernel Compile #62-Ubuntu on
Uptime: .08 days
Memory: 6003456 kB Free:   : 100
Swap: 6180860 kB Free:   : 100

Load Avderage:0.63 0.69 0.61 
Process Count:  total 1 running sleeping 0 stopped 0 zombie
\end{verbatim}
}
\normalsize
